\documentclass[12pt, spanish]{article}
\usepackage[spanish]{babel}
\selectlanguage{spanish}
\usepackage{natbib}
\usepackage{url}
\usepackage[utf8x]{inputenc}
\usepackage{graphicx}
\graphicspath{{images/}}
\usepackage{parskip}
\usepackage{fancyhdr}
\usepackage{vmargin}

\usepackage[default]{sourcesanspro}

\setmarginsrb{2 cm}{1 cm}{2 cm}{2 cm}{1 cm}{1.5 cm}{1 cm}{1.5 cm}

\title{Análisis de Empresa: \\
Twitter, Inc.}                           
\author{Jose Luis Gallego Peña \\
María Sánchez Marcos \\
Antonio David Villegas Yeguas}                             
\date{\today}                                           

\renewcommand*\contentsname{hola}

\makeatletter
\let\thetitle\@title
\let\theauthor\@author
\let\thedate\@date
\makeatother

\pagestyle{fancy}
\fancyhf{}
\rhead{\theauthor}
\lhead{\thetitle}
\cfoot{\thepage}

\begin{document}
%%%%%%%%%%%%%%%%%%%%%%%%%%%%%%%%%%%%%%%%%%%%%%%%%%%%%%%%%%%%%%%%%%%%%%%%%%%%%%%%%%%%%%%%%

\begin{titlepage}
    \centering
    \vspace*{0.5 cm}
    \includegraphics[scale = 0.50]{ugr.png}\\[1.0 cm]
    %\textsc{\LARGE Universidad de Granada}\\[2.0 cm]   
    \textsc{\large 1ºA}\\[0.5 cm]            
    \textsc{\large Grado en Ingeniería Informática}\\[0.5 cm]              
    \rule{\linewidth}{0.2 mm} \\[0.4 cm]
    { \huge \bfseries \thetitle}\\
    \rule{\linewidth}{0.2 mm} \\[1.5 cm]
    
    \begin{minipage}{0.4\textwidth}
        \begin{flushleft} \large
            \emph{Autores:}\\
            \theauthor
            \end{flushleft}
            \end{minipage}~
            \begin{minipage}{0.4\textwidth}
            \begin{flushright} \large
            \emph{Asignatura: \\
            Ingeniería, Empresa y Sociedad}                   
        \end{flushright}
    \end{minipage}\\[1 cm]
  	
    {\large \thedate}\\[1 cm]
 	
    \vfill
    
\end{titlepage}

%%%%%%%%%%%%%%%%%%%%%%%%%%%%%%%%%%%%%%%%%%%%%%%%%%%%%%%%%%%%%%%%%%%%%%%%%%%%%%%%%%%%%%%%%

\tableofcontents
\pagebreak

%%%%%%%%%%%%%%%%%%%%%%%%%%%%%%%%%%%%%%%%%%%%%%%%%%%%%%%%%%%%%%%%%%%%%%%%%%%%%%%%%%%%%%%%%

\section{Introducción}

Twitter es un servicio de \textit{microblogging}, con sede en San Francisco, California, con filiales en San Antonio (Texas) y Boston (Massachusetts) en Estados Unidos. Twitter, Inc. fue creado originalmente en California, pero está bajo la jurisdicción de Delaware desde 2007. 

Fue creado por Jack Dorsey en marzo de 2006, y lanzado en julio del mismo año. Desde entonces la red ha ganado popularidad mundial y se estima que tiene más de 500 millones de usuarios, generando 65 millones de tuits al día y maneja más de 800.000 peticiones de búsqueda diarias. Ha sido denominado como el "SMS de Internet".

Por tanto, en este trabajo nos centraremos en analizar \textbf{\textit{Twitter, Inc.}} como empresa, usando para ello los conocimientos aprendidos en la asignatura \textit{Ingeniería, Empresa y Sociedad} del primer curso del Grado en Ingeniería Informática por la UGR.


\begin{table}[!htb]
\centering
\begin{tabular}{l|r}
Item & Quantity \\\hline
Cosa1 & 99 \\
Cosa2 & 10
\end{tabular}
\caption{\label{tab:widgets}Tabla de ejemplo}
\end{table}

\newpage

\section{Apartado}

\begin{figure}[!htb]
\centering
\includegraphics[width=0.5\textwidth]{ugr.png}
\caption{\label{fig:frog}Esto es el logo de la ugr}
\end{figure}

\newpage


\section{Parte Pepe}

Parte Pepe       %Temas 1, 2 y 3 

\section{La organización en Twitter, Inc.}

La función de organización se basa, principalmente, en dividir el trabajo, tanto humano como material, para posteriormente coordinar las distintas tareas de forma agrupada para la correcta ejecución de los planes establecidos. Para esta organización debemos tener en cuenta distintos factores, como la misión, los objetivos de la empresa, el uso de herramientas para la construcción de esta estructura, entre otros.

En la empresa que analizamos, Twitter, Inc., ya hemos mencionado anteriormente algunos de estos (como la misión).

\subsection{Principios organizativos fundamentales.}

\subsubsection{Principio de la división del trabajo.}

Siguiendo estos principio, Twitter, Inc., se distribuye en distintos equipos de trabajo, entre los que se encuentran:

\textbf{Construir el producto}

El grupo \textit{Build the product} comprende los siguientes equipos:

\begin{itemize}

\item \textit{Data Science and Analytics}
\item \textit{Infraestructure Engineering}
\item \textit{Product}
\item \textit{Software Engineering}
\item \textit{User Services}
\item \textit{Design and Research}

\end{itemize}

\textbf{Grupo: Seguir manteniéndonos}

El grupo \textit{Keep us running} comprende los siguientes equipos:

\begin{itemize}

\item \textit{Finance}
\item \textit{Legal and Public Policy}
\item \textit{People}
\item \textit{Workplace}

\end{itemize}

\textbf{Grupo: Promover el negocio}

El grupo \textit{Promote the business} comprende los siguientes equipos:

\begin{itemize}

\item \textit{Marketing and Communications}
\item \textit{Sales and Partnerships}

\end{itemize}

\textbf{Grupo: Nuestras marcas}

El grupo \textit{Our family of brands} comprende los siguientes equipos:

\begin{itemize}

\item \textit{Periscope}
\item \textit{MoPub}

\end{itemize}


Todos estos equipos se distribuyen entre 35 países alrededor del mundo.

\subsubsection{Principio de la especialización del trabajo.}

Como vemos en el apartado anterior, existen grandes distinciones entre los distintos equipos de trabajo, por lo que es necesaria una especialización, que dependerá de cada equipo, encontrando así un alto nivel de especialización en cada uno de ellos.

\subsubsection{Principio de jerarquía, unidad de mando y ámbito de control.}

Estos principios, explicados y revisados en apartados anteriores, suponen una correcta organización del trabajo, esencial para el desarrollo de los objetivos de la empresa.

\subsubsection{Principio de descentralización.}

Como hemos visto, en Twitter, Inc. vemos una clara descentralización con los distintos grupos de trabajo, sin obviar las decisiones tomadas por los altos directivos. %Temas 4, 5, 6 y 7

\section{Parte Mari}

Parte Mari       %Tema 8, 9 y 10 


\bibliographystyle{plain}
\bibliography{biblist}

\end{document}