

\section{Recursos Humanos en Twitter}

A grandes rasgos, los recursos humanos de una empresa son el conjunto de empleados y colaboradores que
trabajan en la empresa, aunque comunmente nos referimos a ellos como el proceso de reclutamiento, selección, formación, evaluación, compensación... de una empresa. Normalmente, estas políticas son llevadas a cabo por un departamento en específico dentro de la empresa.

\subsection{Reclutamiento}

En Twitter, el reclutamiento es un proceso muy sencillo, ya que en su misma página principal te dan la opción de presentarte para poder trabajar con ellos en sus diferentes departamentos en sucursales de todo el mundo.

\includegraphics[scale=0.25]{reclutamiento1.png}
\newpage

Una vez accedida a esta sección, decides a qué departamento quieres aplicar para pasar al proceso de selección.

\includegraphics[scale=0.25]{reclutamiento2.png}

\subsection{Selección}

El proceso de selección en twitter consta de 3 pasos

\subsubsection{Paso 1}
Después de presentar la solicitud, un reclutador podría comunicarse con el solicitante para efectuar una llamada de presentación.

\subsubsection{Paso 2}
Si es del agrado del reclutador, más tarde le llamarán para entrevistas con una o dos personas más.

\subsubsection{Paso 3}
Si continua a lo largo del proceso, irá a la oficina de Twitter (En el lugar solicitado por el solicitante) una o dos veces para entrevistarse finalmente con 5-10 personas más.

\subsection{Compensación}

Twitter se preocupa del estado en el que sus empleados trabajan, por tanto, para retener el talento, se centran en dos puntos: el aprendizaje y la directiva.
En la empresa tienen organizaciones de aprendizaje de talla mundial enfocadas a Twitter. Tienen Twitter University, enfocado al desarrollo de capacidades de ingeniería móvil para los empleados de la compañía. Por otro lado, creen que la clave para la comodidad de sus empleados es que sus directores sean personas competentes y agradables, por ello, los directores de departamenro tienen 6 horas de clase