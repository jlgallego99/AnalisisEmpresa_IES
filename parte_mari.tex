

\section{Recursos Humanos en Twitter, Inc}

A grandes rasgos, los recursos humanos de una empresa son el conjunto de empleados y colaboradores que
trabajan en la empresa, aunque comunmente nos referimos a ellos como el proceso de reclutamiento, selección, formación, evaluación, compensación... de una empresa. Normalmente, estas políticas son llevadas a cabo por un departamento en específico dentro de la empresa.

\subsection{Reclutamiento}

En Twitter, el reclutamiento es un proceso muy sencillo, ya que en su misma página principal te dan la opción de presentarte para poder trabajar con ellos en sus diferentes departamentos en sucursales de todo el mundo.

\includegraphics[scale=0.25]{reclutamiento1.png}
\newpage

Una vez accedida a esta sección, decides a qué departamento quieres aplicar para pasar al proceso de selección.

\includegraphics[scale=0.25]{reclutamiento2.png}

\subsection{Selección}

El proceso de selección en twitter consta de 3 pasos

\subsubsection{Paso 1}
Después de presentar la solicitud, un reclutador podría comunicarse con el solicitante para efectuar una llamada de presentación.

\subsubsection{Paso 2}
Si es del agrado del reclutador, más tarde le llamarán para entrevistas con una o dos personas más.

\subsubsection{Paso 3}
Si continua a lo largo del proceso, irá a la oficina de Twitter (En el lugar solicitado por el solicitante) una o dos veces para entrevistarse finalmente con 5-10 personas más.

\subsection{Compensación}

Twitter se preocupa del estado en el que sus empleados trabajan, por tanto, para retener el talento, se centran en dos puntos: el aprendizaje y la directiva.
En la empresa tienen organizaciones de aprendizaje de talla mundial enfocadas a Twitter. Tienen Twitter University, enfocado al desarrollo de capacidades de ingeniería móvil para los empleados de la compañía. Por otro lado, creen que la clave para la comodidad de sus empleados es que sus directores sean personas competentes y agradables, por ello, los directores de departamento reciben 6 horas de clase de dirección y motivación al trimestre, de esta manera, los directores estarán implicados y apasionados realmente en el proyecto. En Twitter Inc piensan que no se despiden puestos, sino empleados.

\subsection{Formación}

Como hemos comentado anteriormente, existe una formación interna tanto como para empleados, como para directivos llamada Twitter University.

\subsection{Socialización}

Para facilitar el proceso de adaptación de un nuevo trabajador a la empresa, Twitter ofrece en sus oficinas lugares de ocio como salones y comedores dondde los empleados pueden relacionarse y descansar tras una jornada de trabajo.

\subsection{Relaciones Laborales}

La directora de Recursos Humanos afirma que la clave del éxito en este ámbito es, sin duda, el apoyo que brindan los directores los empleados de sus proyectos, por tanto, hay una estrecha relación entre todos los trabajadores de la empresa.

\section{El Marketing de Twitter, Inc}

Definimos Marketing como: $"$Conjunto de técnicas y estudios que tienen como objeto mejorar la comercialización de un producto$"$.

A continuación explicaremos cómo aborda Twitter las actividades de Marketing.

\subsection{Análisis de Mercado y comportamiento del consumidor}

Twitter se encuentra en un sector muy competitivo: las redes sociales.

Existen muchas aplicaciones de comunicación de este tipo, como Facebook, Instagram, Snapchat, Tumblr... ¿Pero qué es lo que ha hecho a Twitter destacar ante sus competidores?

Twitter se dirige a un mercado joven y dinámico, pero también con el paso del tiempo, se ha transformado en una herramienta para las empresas, tanto para promocionarse y anunciarse como para investigar el comportamiento y preferencias de su clientela.
La clave de su éxito reside en la facilidad de uso que tiene (tanto para publicar tu propio contenido como para interactuar con el contenido ajeno), la posibilidad de personalización, el uso de los llamados \textit{hashtags} para agrupar contenidos, la ausencia de censura exceptuando 5 países y, para atraer visitas y ver de qué se habla en el momento, las \textit{tendencias}.

Nuestra empresa ha sabido localizar el centro de su público y darles las herramientas necesarias para expresarse libremente y de forma segura.

\subsection{Estrategia de Marketing}

La nueva estrategia de Marketing de Twitter incluye vídeos explicando los valores que lo distinguen de las otras plataformas. Este esfuerzo se deriva de la investigación que la compañía reveló. Mientras que el 90\% de las personas reconocían globalmente la marca Twitter, no lo usaban porque no entendían qué era Twitter. La compañía busca ahora cambiar la percepción de que Twitter es una red social qu permite a los usuarios conectarse con familiares y amigos para convertirse en un lugar activo para descubrir noticias.

Twitter se está posicionando como plataforma única donde se pueden acceder tanto a noticias de grandes eventos, hasta noticias locales junto con comentarios en directo.

Esta estrategia que destaca su singularidad atraerá a más público, afirma la directora de Marketing.

También están haciendo convenios con compañías como Samsung para que su aplicación venga instalada de forma predeterminada en sus dispositivos.