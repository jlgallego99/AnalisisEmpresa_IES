\section{La empresa: Elementos y funciones}

La empresa es una organización que transforma un conjunto de recursos físicos, monetarios y cognitivos en bienes y/o servicios, con el objetivo principal de obtener beneficios \cite{}
Atendiendo a esta definición podemos efectivamente definir a Twitter como una empresa, y por tanto en los siguientes subapartados vamos a clasificarla atendiendo a varios criterios:

\subsection{Tamaño}

Según el tamaño tenemos que Twitter es una empresa grande ya que, según datos de 2017, tiene un total de 3.372 trabajadores y un total de ingresos de 7 mil millones y medio de dólares.

\subsection{Sector}

Al tratarse de una empresa que ofrece servicios al público general, y ni extrae materias primas ni las transforma, podemos encasillarla en el sector terciario o sector servicios.

\subsection{Ámbito geográfico de actividad}

Su sede está en San Francisco (California), sin embargo opera en todo el mundo y ofrece gran variedad de idiomas para que cualquier persona del mundo pueda acceder a ella, por tanto estamos hablando de una empresa internacional.

\subsection{Forma jurídica}

Twitter tiene la estructura de Sociedad Anónima, es decir, está estructurada en acciones que cualquiera puede comprar o invertir en ella.

\subsection{Procedencia del capital}

Desde el año 2013 (5 años después de que la red se lanzase) Twitter es una empresa de capital abierto debido a la realización de una Oferta Pública Inicial, sin embargo es como tal una empresa privada ya que el capital es privado, pertenece a accionistas y no al estado.

\subsubsection{Propiedad}

La estructura de propiedad se relaciona con el modo en que se distribuye el capital de las empresas entre sus propietarios legales (accionistas en este caso). La mayor parte de grupos de propiedad en Twitter son de capital industrial o empresarial, en concreto de capital riesgo, centradas en financiar a empresas con alto potencial de crecimiento, pero también hay gran presencia de particulares. Tenemos los siguientes accionistas principales:

\begin{itemize}

\item Kleiner Perkins Caufield and Bayers
\item Benchmark
\item Spark Capital
\item Insight Venture Partners
\item Union Square Ventures
\item Institutional Venture Partners
\item DST Global
\item Alwaleed Bin Talal

\end{itemize}

\section{Dirección y gobierno de la empresa: funciones y niveles}

La función de dirección consiste en gestionar todo el funcionamiento de la empresa para conseguir alcanzar los objetivos empresariales. Para ello es imprescindible que los directivos, personas al cargo de cada tarea y con autoridad, tengan además las capacidades de motivación, liderazgo y comunicación.

\subsection{Dirección ejecutiva}

Omid R. Kordestani es el presidente ejecutivo de Twitter desde 2015.

El director ejecutivo (CEO) de Twitter es Jack Dorsey, siendo también co-fundador de esta. 
La función principal de la dirección ejecutiva es actuar como cabeza visible de la empresa, informando sobre los objetivos, logros o participación de la empresa, así como gestionando la organización y los empleados.

El resto del equipo de liderazgo consta de los siguientes cargos:

\begin{itemize}

\item Parag Agrawal - Director de Recursos Tecnológicos

Está a cargo de dirigir la estrategia empresarial en materia de tecnología y supervisar las áreas de aprendizaje automático e inteligencia artificial en los equipos de productos para usuarios finales, productos rentabilizables y ciencia. Desde su incorporación a Twitter en 2011, Parag ha dirigido las actividades de ampliación de los sistemas de anuncios y ha dado un nuevo impulso al crecimiento de la base de usuarios al mejorar la relevancia de la cronología de inicio.  

\item Ned Segal - Director ejecutivo de Finanzas

Tiene a su cargo la supervisión de las áreas de la empresa relacionadas con finanzas, contabilidad, desarrollo corporativo, seguridad corporativa, análisis y planificación financieros (FPA), relaciones con los inversores, propiedades inmobiliarias e instalaciones corporativas, auditorías internas, procesos tributarios y tesorería.

\item Leslie Berland - Directora de Markting y jefa de Recursos Humanos

Es responsable de las comunicaciones y el marketing de ventas, productos y consumidores a nivel mundial de Twitter. Como jefa de Recursos Humanos, Leslie lidera los equipos de Recursos Humanos, Contrataciones, y Desarrollo Organizacional y de Aprendizaje de Twitter.

\item Vijaya Gadde - Jefa de Asuntos Legales, Política y Seguridad

Maneja los temas de asuntos legales, políticas públicas, y seguridad y privacidad de Twitter.

\item Kayvon Beykpour - Gerente general, Vídeo

Encabeza los productos de video, cargo en el que supervisa los videos, el contenido en vivo y Periscope, una plataforma de videos en vivo que cofundó, donde las personas pueden crear, mirar, descubrir y compartir videos en vivo.

\item Ed Ho - Gerente general, Productos para el Consumidor e Ingeniería

Dirige al equipo responsable de desarrollar y operar el producto de Twitter.

\item Bruce Falck - Gerente general, Productos Rentabilizables e Ingeniería

Supervisa la estrategia e implementación de productos publicitarios para especialistas en marketing en Twitter.

\item Mat Derella - Vicepresidente de Ingresos y Asociaciones

Supervisa las organizaciones de ingresos publicitarios de la empresa, incluidas las ventas publicitarias globales, las asociaciones de contenido globales, el contenido en directo y las operaciones de generación de ingresos.

\item Grace Kim - Vicepresidenta de Investigación de Usuarios y Diseño

Dirige las iniciativas de investigación y diseño de productos de consumo y comerciales.

\end{itemize}

\subsection{Dirección de primera línea}

En el caso de Twitter, Inc. la función de dirección comprende además varios equipos de primera línea con distintas funciones. Podemos distinguir tres grupos diferenciados, cada uno con varios equipos: \textit{Build the product}, \textit{Keep us running} y \textit{Promote the business}

\section{Análisis DAFO}